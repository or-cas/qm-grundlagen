%% required MATH packages
% \usepackage{amsmath,amssymb,amsthm}
% \usepackage{bbm}

% Spin (physical quantity)
\newcommand{\Spin}{\ensuremath{\mathbf{S}}} \newcommand{\spin}{\Spin}
\newcommand{\SpinMagnitude}{\ensuremath{S}} \newcommand{\spinm}{\SpinMagnitude}

% magnetic moment (physical quantity)
\newcommand{\MagneticMoment}{\ensuremath{\boldsymbol{\mu}}} \newcommand{\magnmom}{\MagneticMoment}
\newcommand{\MagneticMomentMagnitude}{\ensuremath{\mu}} \newcommand{\magnmomm}{\MagneticMomentMagnitude}

% Bohr magneton
\newcommand{\BohrMagneton}{\ensuremath{\mu_B}} \newcommand{\bohrmag}{\BohrMagneton}

% magnetic field
\newcommand{\MagneticField}{\ensuremath{\mathbf{B}}} \newcommand{\magnf}{\MagneticField}
\newcommand{\MagneticFieldMagnitude}{\ensuremath{B}} \newcommand{\magnfm}{\MagneticFieldMagnitude}
% h-bar / 2
\newcommand{\hbarh}{\ensuremath{\frac{\hbar}{2}}}

% Stern Gerlach apparatus
\newcommand{\SternGerlach}{\ensuremath{\text{SG}}} \newcommand{\sg}{\SternGerlach}

% Ket
\let\origket\ket % avoids recursion in renewcommand!
\renewcommand{\ket}[1]{\ensuremath{{\origket{#1}}}} % reduces space

% Bra
\let\origbra\bra % avoids recursion in renewcommand!
\renewcommand{\bra}[1]{\ensuremath{{\origbra{#1}}}} % reduces space

% S_z + state
\newcommand{\SpinZUpState}{\ensuremath{\ket{+}}} \newcommand{\szus}{\SpinZUpState}
% S_z - state
\newcommand{\SpinZDownState}{\ensuremath{\ket{-}}} \newcommand{\szds}{\SpinZDownState}
% S_z +/- state
\newcommand{\SpinZUpDownState}{\ensuremath{\ket{\pm}}} \newcommand{\szs}{\SpinZUpDownState}

% hilbert space
\newcommand{\HilbertSpace}{\ensuremath{\mathcal{H}}} \newcommand{\hs}{\HilbertSpace}

% Komplexe Zahlen
\newcommand{\ComplexNumbers}{\ensuremath{\mathbbm{C}}} \newcommand{\comps}{\ComplexNumbers}

% Operator
\newcommand{\Operator}[1]{\ensuremath{\hat{#1}}} \newcommand{\op}[1]{\Operator{#1}}

% S_z operator
\newcommand{\SpinZOperator}{\ensuremath{\op{\SpinMagnitude}_z}} \newcommand{\szop}{\SpinZOperator}

% Dualraum
\newcommand{\DualSpace}[1][\HilbertSpace]{\ensuremath{{#1}^*}} \newcommand{\ds}[1][\HilbertSpace]{\DualSpace[#1]}

% Duale Korrespondenz
\newcommand{\DualCorrespondence}{\ensuremath{\overset{\text{DC}}{\leftrightarrow}}} \newcommand{\dc}{\DualCorrespondence}

% alpha ket
\newcommand{\AlphaKet}{\ensuremath{\ket{\alpha}}} \newcommand{\keta}{\AlphaKet}
\newcommand{\AlphaKetCoefficient}{\ensuremath{c_{\alpha}}} \newcommand{\ca}{\AlphaKetCoefficient}

% alpha bra
\newcommand{\AlphaBra}{\ensuremath{\bra{\alpha}}} \newcommand{\braa}{\AlphaBra}

% beta ket
\newcommand{\BetaKet}{\ensuremath{\ket{\beta}}} \newcommand{\ketb}{\BetaKet}
\newcommand{\BetaKetCoefficient}{\ensuremath{c_{\beta}}} \newcommand{\cb}{\BetaKetCoefficient}

% beta bra
\newcommand{\BetaBra}{\ensuremath{\bra{\beta}}} \newcommand{\brab}{\BetaBra}

% complex conjugate
\newcommand{\ComplexConjugate}[1]{\ensuremath{{#1}^*}} \newcommand{\cc}[1]{\ComplexConjugate{#1}}

% X Operator
\newcommand{\OperatorX}{\ensuremath{\Operator{X}}} \newcommand{\opx}{\OperatorX}
% Y Operator
\newcommand{\OperatorY}{\ensuremath{\Operator{Y}}} \newcommand{\opy}{\OperatorY}

% hermitian adjoint
\newcommand{\Hermitian}[1]{\ensuremath{{#1}^{\dag}}}

\newcommand{\OperatorXHermitian}{\ensuremath{\Hermitian{\OperatorX}}} \newcommand{\opxh}{\OperatorXHermitian}

% Operator x ket
% \newcommand{\OperatorKet}[2]{\ensuremath{{#1}\negthinspace{#2}}} \newcommand{\opk}[2]{\OperatorKet{#1}{#2}}
% bra x Operator
% \newcommand{\BraOperator}[2]{\ensuremath{{#1}{#2}}} \newcommand{\bop}[2]{\BraOperator{#1}{#2}}
