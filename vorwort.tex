Das vorliegende Dokument entsteht als Skript f\"ur eine einw\"ochige Lehrveranstaltung \enquote{Repetitorium Quantenmechanik}  der Verfasser in den Sommersemestern 2011 und 2012 an der Fakult\"at f\"ur Physik der Universit\"at G\"ottingen.

Ziel des Repetitoriums und dieses Skriptes ist, innerhalb von f\"unf Tagen die zentralen theoretischen Inhalte der Pflichtvorlesung \enquote{Quantenmechanik} im Bachelorstudiengang Physik zu wiederholen und anhand von \"Ubungsaufgaben aufzuarbeiten. Unser Schwerpunkt bei Auswahl und Darstellung des Stoffes ist, insbesondere f\"ur die nicht theoretisch oder mathematisch veranlagten Teilnehmer in der gegebenen Zeit einen roten Faden durch Begriffe und vor allem Rechenmethoden der Quantenmechanik zu spinnen. Daf\"ur m\"ussen wir an vielen Stellen auf weiterf\"uhrende Darstellungen verzichten, die \"uber das aus unserer Sicht notwendige Verstehen zum Bestehen hinausgehen, obgleich sie Vorlesungsstoff und damit auch klausurrelevant sein m\"ogen.

Wir danken unseren Vorg\"angern Christian K\"ohler, Maria Lenius, Rene Schulz, Christoph Solveen und Fabian Stiewe f\"ur das Erstellen und \"Uberlassen ihrer Unterlagen f\"ur das Repetitorium Quantenmechanik, die uns wertvolle Anregungen zur Strukturierung und Ausgestaltung unseres Skripts sowie \"Ubungsaufgaben geliefert haben. Die in diesem Skript gemachten Fehler sind nat\"urlich unsere eigenen.