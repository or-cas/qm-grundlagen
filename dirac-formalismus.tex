\subsection{Der quantenmechanische Zustand (ket-Vektoren)}

Der sequentielle Stern-Gerlach-Versuch widerlegt unsere klassische Intuition: physikalische Messungen ver\"andern n\"amlich im Allgemeinen den Zustand unseres Systems! Ein weiterer Widerspruch zu unserer klassischen Intuition ist, dass wir nicht alle physikalischen Gr\"o\ss{}en gleichzeitig messen k\"onnen. Wir k\"onnen daher nicht alles gleichzeitig \"uber einen Zustand wissen.

Wir definieren daher nach \textcite{Nolting}:

\begin{post}
 Gleichzeitige Messung eines maximalen Satzes von \enquote{vertr\"aglichen}, d.h. simultan messbaren, Eigenschaften \enquote{pr\"apariert} einen \enquote{reinen} quantenmechanischen Zustand.
\end{post}

\begin{bsp}
 Bez\"uglich des Elektronenspins haben wir bereits (reine) Zust\"ande kennengelernt: \szus\ und \szds.
\end{bsp}

Wie k\"onnen wir nun mit diesen abstrakten quantenmechanischen Zust\"anden rechnen, also konkrete physikalische Vorhersagen treffen?

\begin{post}
 In der Quantenmechanik wird ein physikalisches System durch einen \emph{Zustandsvektor} \keta\ in einem komplexen Vektorraum, einem sogenannten Hilbertraum \hs, repr\"asentiert. \keta\ enth\"alt die gesamte Information \"uber den physikalischen Zustand. 
\end{post}

Nach Dirac nennen wir einen Zustandsvektor \keta\ auch \emph{ket-Vektor} (kurz: \emph{ket}), und den Hilbertraum \emph{ket-Raum}.

Wie Vektoren k\"onnen wir kets $\keta, \ketb \in \hs$ addieren, die Summe ist wieder ein ket:
\begin{equation*}
 \keta + \ketb = \ket{\gamma} \in \hs
\end{equation*}
Wie ein Vektor l\"asst sich auch ein ket mit einer komplexen Zahl $c \in \comps$ multiplizieren, und wir erhalten wieder ein ket:
\begin{equation*}
 c \keta = \keta c \in \hs
\end{equation*}

\begin{post}
 $\keta$ und $c \keta$ mit $c \neq 0$ sind zwar zwei verschiedene Elemente im ket-Raum \hs, repr\"asentieren aber den selben physikalischen Zustand.
\end{post}

\subsection{Observablen}
Wie k\"onnen wir nun konkret etwas \"uber unser physikalisches System erfahren, das durch einen abstrakten Zustandsvektor repr\"asentiert wird? Durch messbare physikalische Gr\"o\ss{}en:

\begin{post}
 Messbare physikalische Gr\"o\ss{}en (\emph{Observablen}) werden durch lineare Abbildungen (\emph{Operatoren}) auf dem ket-Raum repr\"asentiert.
\end{post}
Eine Observable $A$ wird also repr\"asentiert durch den Operator $\op{A}: \hs \rightarrow \hs$, $\op{A}( \keta) \equiv \op{A}\keta \equiv A \keta \in \hs$ ist wieder ein ket.
\begin{konv}
 Mit der Schreibweise $A$ ist oft der Operator $\op{A}$ der Observablen $A$ gemeint!
\end{konv}
Im Allgemeinen ist das ket $\op{A} \keta$ nicht gleich einer komplexen Zahl multipliziert mit $\keta$! Aber:
\begin{defn}
Die Eigenvektoren $\ket{a_i}$ der linearen Abbildung $\op{A}$ bezeichnen wir auch als \emph{Eigenkets} oder \emph{Eigenzust\"ande} des physikalischen Systems bez\"uglich der Observablen $A$, mit den jeweiligen Eigenwerten $a_i \in \comps$:
\begin{equation*}
 \op{A} \ket{a_i} = a_i \ket{a_i}
\end{equation*}
\end{defn}

\begin{bsp}
 Spin-$\frac{1}{2}$-Systeme haben bez\"uglich der $z$-Richtung die Eigenvektoren \szus\ und \szds\ mit den Eigenwerten $+\hbarh$ und $-\hbarh$:
 \begin{align*}
  \szop \szus &= + \hbarh \szus \\
  \szop \szds &= - \hbarh \szds \\
  \text{oder auch:} \quad \szop \szs &= \pm \hbarh \szs
 \end{align*}

\end{bsp}

\begin{post}
 Gegeben sei eine Observable $A$ mit den Eigenzust\"anden $\ket{a_i}$. Wir k\"onnen jeden Zustandsvektor $\keta$ nach diesen Eigenzust\"anden entwickeln, sprich als Summe der Eigenzust\"ande mit komplexen Vorfaktoren $c_i \in \comps$ schreiben:
\begin{equation}
 \keta = \sum_i c_i \ket{a_i}
\end{equation}
Wir sagen auch: Die Eigenzust\"ande $\ket{a_i}$ spannen den Ketraum $\hs$ auf.
\end{post}

\subsection{Duale Zustandsvektoren (bras) und Skalarprodukt}
\begin{post}
 Zu jedem Zustandsvektor $\keta \in \hs$ existiert ein \emph{duales} ket $\braa \in \ds{\hs}$ im Dualraum von \hs, dem Raum \ds{\hs}\ der linearen Abbildungen $\hs \rightarrow \comps$. Wir bezeichnen $\braa$ auch als das zum ket $\keta$ korrespondierende \emph{bra} (oder den \emph{bra-Vektor}), und \ds{\hs}\ als \emph{bra-Raum}, und schreiben:
\begin{equation}
 \keta \dc \braa
\end{equation}
 Wir postulieren weiterhin, dass wir mit den korrespondierenden bras wie mit den kets rechnen k\"onnen:
\begin{equation}
 c_{\alpha} \keta + c_{\beta} \ketb \dc \cc{c_{\alpha}} \braa + \cc{c_{\beta}} ,
\end{equation}
wobei die Vorfaktoren durch duale Korrespondenz komplex konjugiert werden!
\end{post}

\begin{post}
 Um die Dirac'sche bra-ket-Notation zu vollenden, postulieren wir nun noch das \emph{innere Produkt} (Skalarprodukt) $\braket{\beta | \alpha} \in \comps$ eines bras $\brab$ und eines kets $\keta$:
\begin{equation}
 \braket{\beta | \alpha} \equiv \brab \cdot \keta \equiv \brab ( \keta ) \in \comps
\end{equation}
Das innere Produkt habe folgende Eigenschaften:
\begin{align}
 \braket{\beta | \alpha} &= \cc{\braket{\alpha | \beta}} \\
\braket{\alpha | \alpha} &\geq 0 \qquad \text{\enquote{positiv definit}} 
\end{align}

\end{post}
 
\begin{defn}
 Wir nennen zwei physikalische Zust\"ande bzw. ihre repr\"asentierenden Zustandsvektoren \keta, \ketb, \emph{orthogonal}, wenn gilt $\braket{\beta | \alpha} = 0$.
\end{defn}
\begin{notiz}
 \begin{equation}
  \braket{\beta | \alpha} = 0 \Leftrightarrow \braket{\alpha | \beta} = 0
 \end{equation}
\end{notiz}

\begin{defn}
 Wir nennen einen Zustandsvektor \keta\ \emph{normiert}, wenn $\braket{\alpha | \alpha} = 1$ gilt.
\end{defn}
\begin{konv}
 Da ein physikalischer Zustand von einem ket nur bis auf einen komplexen Vorfaktor festgelegt ist, k\"onnen wir festlegen, dass wir im folgenden nur noch normierte Zustandsvektoren betrachten.
\end{konv}

\subsection{Operatoren}
Wenden wir uns nun wieder den Operatoren zu, und zwar ganz allgemein linearen Abbildungen auf dem Ketraum \hs, nicht nur solchen, die physikalisch messbaren Gr\"o\ss{}en (Observablen) entsprechen.

\begin{eig}
Auf ein ket \keta\ wirkt ein Operator \opx\ ganz normal \enquote{von links}, d.h. $\opx(\keta) \equiv \opx \keta \in \hs$ ist wieder ein ket.
\end{eig}

\begin{eig}
 Wie Funktionen auf den reellen Zahlen auch werden Operatoren dar\"uber definiert, wie sie auf (alle) kets wirken.
\end{eig}

\begin{bsp}
 Zwei Operatoren $\opx, \opy$ sind gleich, $\opx=\opy$, wenn $\opx \keta = \opy \keta$ f\"ur alle kets $\keta$ gilt.
\end{bsp}

\begin{eig}
Operatoren lassen sich \enquote{wie Zahlen} zu neuen Operatoren addieren, kommutativ und assoziativ:
\begin{align}
\opx + \opy &= \opy + \opx \\
\opx + (\opy + \op{Z}) &= (\opx + \opy) + \op{Z}
\end{align}
\end{eig}

\begin{eig}
 Operatoren sind linear:
 \begin{equation}
  \opx(\ca \keta + \cb \ketb) = \ca \opx \keta + \cb \opx \ketb
 \end{equation}
\end{eig}

\begin{eig}
Operatoren wirken auch auf bras, und zwar \emph{von rechts}!
\begin{equation}
\opx(\braa) \equiv \braa \opx \in \ds
\end{equation} 
$\braa\opx$ ist wieder ein bra!
\end{eig}

\begin{notiz}
 Im Allgemeinen korrespondiert das bra $\braa \opx$ nicht mit dem ket $\opx \keta$! Aber:
\end{notiz}

\begin{defn}
 Sei \opx\ ein Operator. Der zu \opx\ \emph{hermitesch adjungierte Operator} oder auch einfach \emph{adjungierte Operator} \opxh\ sei f\"ur alle kets \keta\ definiert \"uber die duale Korrespondenz
 \begin{equation}
  \opx \keta \dc \braa \opxh
 \end{equation}

\end{defn}

\begin{defn}
 Gilt f\"ur einen Operator $\opxh = \opx$, d.h. $\opx \keta \dc \braa \opx$ f\"ur alle kets \keta, so hei\ss{}t \opx\ \emph{hermitesch}.
\end{defn}

\begin{eig}
 Wir k\"onnen Operatoren multiplizieren, d.h. hintereinander auf einen Zustandsvektor \keta\ anwenden, und wieder ein ket erhalten: $\opy \opx \keta \equiv \opy ( \opx \keta) \in \hs$.
\end{eig}
