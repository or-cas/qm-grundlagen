%\usepackage{bbm}
%\usepackage{color}
%\usepackage{caption}
%\usepackage{psfrag}
%\usepackage{esint}
% \parindent0em
% \parskip1.5ex plus0.5ex minus0.5ex
% \topmargin4mm
% \headheight5mm
% \headsep6mm
% \topskip5mm
% \textwidth170mm
% \evensidemargin5mm
% \oddsidemargin5mm
% \textheight210mm
% \hoffset -10mm
% \footskip20mm

%\renewcommand{\textfraction}{0}
%\renewcommand \thesection {\Roman{section}}
%\newcommand{\eps}{\epsilon}
%\newcommand{\qed}{\begin{flushright}$\square$\end{flushright}}

\section{Die Wellenfunktion}

Die Tatsache, da"s Teilchen Welleneigenschaften haben, legt es nahe, 
Teilchen mit Hilfe von Wellen zu beschreiben. Spezielle Wellen sind
die ebenen Wellen:
\begin{equation}
\label{eq::ebene_welle}
	\Psi(\vec x,t) = A e^{i(\vec k \vec x -\omega t)}
\end{equation}
mit einer Dispersionsrelation $\omega = \omega(\vec k)$
(Beispiel Lichtwelle: $\omega = c |\vec k|$).

Ebene Wellen sind "uberall im Raum verteilt. Um ein lokalisiertes Teilchen
zu beschreiben, mu"s man ebene Wellen zu sogenannten Wellenpaketen
"uberlagern:
\begin{equation}
	\Psi(\vec x,t) = \frac{1}{\sqrt{2\pi}}\int \mathrm{d}^3k
	\hat \Psi(\vec k)
	e^{i(\vec k \vec x -\omega t)} = \frac{1}{\sqrt{2\pi}}
	\int \mathrm{d}^3 k\hat \Psi(\vec k,t) e^{(i \vec k \vec x)}.
\end{equation}
Die Funktionen $\Psi$ und $\hat \Psi$ sind also gerade "uber die
Fouriertransformation miteinander verbunden. 

\section{Materiewellen}

F"ur Materiewellen freier Teilchen
gilt mit $E=\hbar \omega$, $\vec p = \hbar \vec k$ und
$E = \frac{|\vec p|^2}{2m}$) die Dispersionsrelation:
\begin{equation}
	\omega(\vec k) = \frac{\hbar |\vec k|^2}{2m}.
\end{equation}
Ableiten der Funktion $\Psi(\vec x,t)$ liefert allgemein:
\begin{align}
	\frac{\partial }{\partial  t} \Psi(\vec x,t) &= 
	\frac{1}{\sqrt{2\pi}}\int \mathrm{d}^3k
        \hat \Psi(\vec k) (-i\omega)
        e^{i(k \vec x -\omega t)}\\
	\Delta \Psi(\vec x,t) &= \frac{1}{\sqrt{2\pi}}\int \mathrm{d}^3k
        \hat \Psi(\vec k) (-|\vec k|^2)
        e^{i(\vec k \vec x -\omega t)}.
\end{align}
Damit gilt f"ur Materiewellen mit obiger Dispersionsrelation:
\begin{equation}
	i\hbar \frac{\partial }{\partial  t} \Psi(\vec x,t)
	= - \frac{\hbar^2}{2m} \Delta \Psi(\vec x,t).
\end{equation}
Die ist die \emph{Schr"odingergleichung f"ur freie Teilchen}.

Was "andert sich f"ur Teilchen in einem Potential $U(\vec x)$?
Es gilt nun $E=\hbar \omega= \frac{\vec p^2}{2m} + U(\vec x)$. Damit
ergibt sich die Schr"odingergleichung:
\begin{equation}
	i\hbar \frac{\partial }{\partial  t} \Psi(\vec x,t)
	= H \Psi(\vec x,t)
\end{equation}
mit dem Hamilton-Operator
\begin{equation}
	H = - \frac{\hbar^2}{2m}\Delta + U(\vec x).
\end{equation}
F"ur L"osungen der Schr"odingergleichung gilt:
\begin{enumerate}
	\item Superpositionsprinzip: Sind $\Psi$, $\phi$ L"osungen,
	dann ist auch $\alpha \Psi + \beta \phi$ mit $\alpha,
	\beta  \in \mathbb{C}$ eine L"osung.
	\item Determiniertheit: Wenn wir $\Psi(\cdot,0)$ kennen, ist
	durch die Schr"odingergleichung die Wellenfunktion zu allen
	Zeiten bestimmt.
	\item Erhaltungssatz: Es gilt $\int \mathrm{d}^3 x
	|\Psi(\vec x,t)|^2 = const.$ (Beweis:
	$\int \mathrm{d}^3 x
        |\Psi(\vec x,t)|^2 = \int \mathrm{d}^3 k
        |\Psi(\vec k,t)|^2 = \int \mathrm{d}^3 k
        |\Psi(\vec k)e^{i\omega t}|^2 = \int \mathrm{d}^3 k
        |\Psi(\vec k)|^2$, wobei wir die Parsevalsche Gleichung verwendet
	haben.)	 
\end{enumerate}

\section{Interpretation der Wellenfunktion als Wahrscheinlichkeitsdichte}

Das Betragsquadrat der normierten Wellenfunktion wird als 
Wahrscheinlichkeitsdichte interpretiert, d.h. $|\Psi(\vec x,t)|^2\mathrm{d}^3x$
gibt die Wahrscheinlichkeit an, das Teilchen zum Zeitpunkt $t$ in einem
Volumenelement $\mathrm{d}^3 t$ am Ort $\vec x$ zu finden. 
Entsprechend ist $|\hat \Psi(\vec k,t)|^2$ die
Wahrscheinlichkeitsdichte f"ur den Wellenvektor (direkter Zusammenhang
zum Impuls: $\vec p = \hbar \vec k$).

Wir betrachten nun die zeitliche Entwicklung der Wahrscheinlichkeitsdichte:
\begin{equation}
\begin{split}
	\frac{\partial }{\partial  t} |\Psi(\vec x,t)\Psi^*(\vec x,t)|^2
	&= \dot \Psi \Psi^* + \Psi \dot \Psi^* =
	-\frac{i}{\hbar} (\hat H\Psi) \Psi^* + \frac{i}{\hbar} 
	\Psi(\hat H\Psi^*)\\
	&= - \frac{i}{\hbar} \left[-\frac{\hbar^2}{2m}(\Delta \Psi)
	+ U(\vec x) \Psi \right] \Psi^* + \frac{i}{\hbar} \Psi
	\left[-\frac{\hbar^2}{2m}(\Delta \Psi^*)+U(\vec x)\Psi^* \right]\\
	&=\frac{\hbar}{2mi} \left[(\Delta \Psi^*)\Psi - (\Delta \Psi)
	\Psi^* \right].
\end{split}
\end{equation}
Mit der Wahrscheinlichkeitsstromdichte $\vec j(\vec x,t) = \frac{\hbar}{2mi}
\left[\Psi^*(\nabla \Psi) - (\nabla \Psi^*)\Psi \right]$ ergibt
sich die Kontinuit"atsgleichung:
\begin{equation}
	\frac{\partial}{\partial t} \rho(\vec x,t) + \nabla \vec j(\vec x,t)=0
\end{equation}
mit
\begin{equation}
	\rho(\vec x,t) = |\Psi(\vec x,t)|^2.
\end{equation}

\section{Quantenmechanische Observablen}

Zun"achst ben"otigen wir einige Definitionen.\\
\textbf{Definition:} Das Skalarprodukt zweier Wellenfunktionen
$\Psi$ und $\phi$ ist gegeben durch
\begin{equation}
	(\phi,\Psi) := \int \mathrm{d}^3 x \phi^*(\vec x)\Psi(\vec x). 
\end{equation}
\textbf{Definition:} $A^{+}$ hei"st zu $A$ adjungierter Operator,
wenn gilt:
\begin{equation}
	(A^+ \phi, \Psi) = (\phi, A \Psi)
\end{equation}
f"ur beliebige $\phi$, $\Psi$.\\
\textbf{Definition:} Der Operator $A$ hei"st hermitisch, wenn $A^+ = A$.
F"ur hermitische Operatoren gilt:
\begin{enumerate}
	\item Eigenwerte hermitischer Operatoren sind reell.
	\item Eigenfunktionen hermitischer Operatoren zu verschiedenen
		Eigenwerten sind orthogonal.
\end{enumerate}
Die Eigenfunktionen der f"ur uns relevanten Operatoren sind vollst"andig,
d.h., sie spannen den ganzen Raum auf.

Observablen werden durch hermitische Operatoren beschrieben. Der Mittelwert
im Zustand $\Psi$ ist gegeben durch
\begin{equation}
	\langle A \rangle = \int \mathrm{d}^3 x \Psi^*(\vec x,t) A \Psi(\vec x,t).
\end{equation}
Die Zeitentwicklung des Mittelwerts wird beschrieben durch:
\begin{equation}
	\frac{\mathrm{d}}{\mathrm{d}t} \langle A \rangle =
	\frac{i}{\hbar} \langle[H,A]\rangle + \langle \frac{\partial A}
	{\partial t}\rangle
\end{equation}
mit dem Kommutator $[B,C] = BC-CB$. Diesen Zusammenhang nennt man
das \emph{Ehrenfestsche Theorem}.

Quantenmechanische Zust"ande $\Psi(\vec x)$ lassen sich nach Eigenfunktionen
hermitischer Operatoren entwickeln, z.B. f"ur ein diskretes Spektrum:
\begin{equation}
	\Psi(\vec x)=\sum \limits_{n} c_n \Psi_n(\vec x) \quad \textnormal{mit}
	\quad c_n = (\Psi_n,\Psi).
\end{equation}
Dabei gilt: $\sum \limits_n |c_n|^2 = 1$. 

F"ur den Mittelwert der Observablen, die durch den Operator $A$ 
beschrieben wird, finden wir:
\begin{equation}
	\langle A \rangle = \sum \limits_{n} |c_n|^2 a_n.
\end{equation}
Interpretation: Mit Wahrscheinlichkeit $|c_n|^2$ messen wir den Wert
$a_n$. Die Wellenfunktion geht dann in den Zustand $\Psi_n$ "uber.

Beispiele f"ur Operatoren (mit kontinuierlichem Spektrum) sind 
der Orts- und der Impulsoperator.\\
\textbf{Ortsoperator (in einer Dimension):} 
\begin{itemize}
	\item Eigenwertgleichung: $x\Psi_{\xi}(x)=\xi \Psi_{\xi}(x)$
	\item Eigenfunktionen: $\Psi_{\xi}(x) = \delta(x-\xi)$.
\end{itemize}
\textbf{Impulsoperator (in einer Dimension):}
\begin{itemize}
	\item Eigenwertgleichung: $\frac{\hbar}{i} \frac{\partial}{\partial x}
		\Psi_p(x) = p \Psi(x)$.
	\item Eigenfunktionen: $\Psi_p(x) = (2 \pi \hbar)^{-1/2}
		e^{ipx/\hbar}$.
\end{itemize}



\section{Die station"are Schr"odingergleichung}

Falls das Potential $U(\vec x)$ zeitunabh"angig ist, kann die 
Schr"odingergleichung mittels eines Separationsansatzes gel"ost werden:
\begin{equation}
	\Psi(\vec x,t) = f(t)\Psi(\vec x).
\end{equation}
Einsetzen dieses Ansatzes leifert: 
\begin{equation}
	\frac{1}{f(t)}i\hbar \frac{\partial}{\partial t} f(t)
	= \frac{1}{\Psi(\vec x)}H\Psi(\vec x).
\end{equation}
Die linke Seite h"angt nur von $t$ ab, die rechte nur von $x$. Beide
Seiten m"ussen also gleich einer Konstanten $E$ sein. Damit ergibt
sich f"ur den zeitabh"angigen Teil:
\begin{equation}
	i\hbar \frac{\partial}{\partial t}f(t)=E f(t)
	\quad \Rightarrow \quad f(t)=e^{-iEt/\hbar}.
\end{equation} 
F"ur den ortsabh"angigen Teil erhalten wir:
\begin{equation}
	H\Psi(\vec x) = E\Psi(\vec x).
\end{equation}
Diese Gleichung nennt man \emph{zeitunabh"angige Schr"odingergleichung}.
(Die Zust"ande $\Psi(\vec x,t) = \exp{(-iEt/\hbar)}\Psi(\vec x)$ 
hei"sen station"are Zust"ande, da gilt: 
$|\Psi(\vec x,t)|^2=|\Psi(\vec x)|^2$.)

Die allgemeine L"osung der Schr"odingergleichung l"a"st sich nun nach
diesen station"aren L"osungen entwickeln:
\begin{equation}
	\Psi(\vec x,t) = \sum \limits_{n} c_n e^{-iE_nt/\hbar}\Psi_n(\vec x)
	\quad \textnormal{mit} \quad c_n = (\Psi_n,\Psi(t=0)).
\end{equation}


\section{Beispiel: Die Potentialstufe}

siehe z.B. Schwabl, Quantenmechanik


